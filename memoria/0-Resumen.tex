\chapter*{Resumen}

Deep learning ha surgido por sus grandes mejoras respecto a las técnicas reinantes en los problemas más complicados en Inteligencia Artificial, inversiones masivas de gigantes industriales y por un crecimiento exponencial en el número de publicaciones científicas. Deep learning es una herramienta más dentro del conjunto de herramientas de Machine Learning, cuyo propósito es hacer aprender a las máquinas.

En ciertas áreas de la visión artificial han sido muy exitosas, sin embargo, en el campo del seguimiento visual aún están por desarrollar, por eso hemos abordado el problema del seguimiento visual de múltiples peatones con técnicas de deep learning. Así, en este trabajo se ha diseñado y construido un componente software que combina \textit{tracking by detection}, empleando técnicas de deep learning, con \textit{tracking by matching}, usando el algoritmo de Lucas-Kanade. Combinando estas dos técnicas, recogemos sus ventajas, minimizando el efecto de sus inconvenientes. Además, el componente, también incorpora un mecanismo de reidentificación de peatones que mejora el seguimiento.

Finalmente, el componente desarrolado se ha validado experimentalmente y se ha probado en la conocida base de datos de seguimiento visual \textit{Multiple object tracking}.

